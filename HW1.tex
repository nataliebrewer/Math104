\documentclass{article}
\usepackage{graphicx} % Required for inserting images
\usepackage{amsmath}
\usepackage{amssymb}
\usepackage{amsthm}


\title{Math 104 Homework 1}
\author{Natalie Brewer}
\date{January 2024}

\begin{document}

\maketitle

\section{Exercises 1.2}
\textbf{1.2.1}
(a) Prove that $\sqrt{3}$ is irrational. Does a similar argument work to show $\sqrt{6}$ is irrational?

\textbf{Proof}: Suppose for contradiction that $\sqrt{3}$ is rational. Let p and q be integers such that  $\frac{p}{q} = \sqrt{3}$ is in least terms. Thus, p and q have no common factors.

    \begin{equation}
        \left( \frac{p}{q} \right)^2 = 3
    \end{equation}
    \begin{equation}
        p^2 = 3q^2
    \end{equation}

So 3 divides $p^2$ and hence divides $p$ because 3 is prime. So $p = 3r$ for some $r \in \mathbb{Z}$.
    \begin{equation}
        p^2 = (3r)^2 = 9r^2 = 3q^2
    \end{equation}
    \begin{equation}
        3r^2 = q^2
    \end{equation}

Now 3 divides $q^2$ and hence $q$. So $p$ and $q$ share a common factor 3, contradicting our initial assumption. \qed\\

A similar argument works to show that $\sqrt{6}$ is irrational. However, instead of arguing that $6 \mid p^2$ implies that $6 \mid p$ because it is prime, we use the fact that 2 and 3 must divide $p$ because they are the prime factors of 6. This implies that $6 \mid p$. \\

(b) Where does the proof of Theorem 1.1.1 break down if we try to use it to prove $\sqrt{4}$ is irrational?

The proof breaks down in the step where we would say $4 \mid p^2$ implies $4 \mid p$. Because 4 is a perfect square, this statement is not necessarily true. Consider the counterexample where $p = 2$. Here, 4 divides $p^2 = 4$, but not $p$.\\



\noindent \textbf{1.2.5} 
Let A and B be subsets of $\mathbb{R}$.

(a) If $x \in (A \cap B)^c$, explain why $x \in A^c \cup B^c$. This shows that $(A \cap B)^c 	\subseteq A^c \cup B^c$.

If x belongs to the complement of the intersection $A \cap B$, then x does not belong to the intersection. So x does not belong to both A and B. This means that x must belong to at least one of $A^c$ or $B^c$. \\

(b) Prove the reverse inclusion $(A \cap B)^c \supseteq A^c \cup B^c$, and conclude that $(A \cap B)^c = A^c \cup B^c$.

\textbf{Proof}: Let $x \in (A^c \cup B^c)$. Then $x \in A^c$ or $x \in B^c$ or both. This means that $x \notin A$ or $x \notin B$ or x is in neither. So $x$ cannot be in both A and B, the intersection. Written using operators, $x \notin (A \cap B)$. This is equivalent to saying $x \in (A \cap B)^c$. \qed\\

(c) Show $(A \cup B)^c = A^c \cap B^c$ by demonstrating inclusion both ways.

\textbf{Proof}: First we show that $x \in (A \cup B)^c \implies x \in A^c \cap B^c$.

Suppose $x \in (A \cup B)^c$. This means that $x$ is not in $A \cup B$. Therefore, $x$ is not in A and $x$ is not in B. Hence, $x \in A^c $ and $x \in B^c$. This implies $x \in A^c \cap B^c$. \\


Next, we show that $x \in A^c \cap B^c \implies x \in (A \cup B)^c$.

Suppose $x \in A^c \cap B^c$. This means that $x$ is not in $A$ and $x$ is not in $B$. Therefore, $x$ is not in $A \cup B$. Hence, $x \in (A \cup B)^c $.

We have shown that each set is a subset of the other and therefore, $$ (A \cup B)^c = A^c \cap B^c $$ \qed\\



\noindent \textbf{1.2.7}
Given a function $f$ and a subset $A$ of its domain, let $f(A)$
represent the range of $f$ over the set $A$; that is, $f(A) = \{f(x) : x \in A\}$.

(a) Let $f(x) = x^2$. If $A = [0, 2]$ (the closed interval $\{x \in \mathbb{R} : 0 \leq x \leq 2\}$) and $B = [1, 4]$, find $f(A)$ and $f(B)$. Does $f(A \cap B) = f(A) \cap f(B)$ in this case? Does $f(A \cup B) = f(A) \cup f(B)$?

$$ f(A) = [0,4]$$
$$ f(B) = [1,16]$$
$$ f(A \cap B) = f([1,2]) = [1,4]$$
$$ f(A) \cap f(B) = f([1,2]) = [0,4] \cap [1, 16] = [1,4]$$

$$ f(A \cup B) = f([0,4]) = [0,16]$$
$$ f(A) \cup f(B) = [0,4] \cup [1,16] = [0,16]$$

In this case $f(A \cap B) = f(A) \cap f(B)$ and $f(A \cup B) = f(A) \cup f(B)$. \\

(b) Find two sets $A$ and $B$ for which $f(A \cap B) \neq f(A) \cap f(B)$.

Let $A = [0,1]$ and $B = \{-1\}$. Then
$$ f(A) = [0,1]$$
$$ f(B) = \{1\} $$
$$ f(A \cap B) = f(\emptyset) = \emptyset$$
$$ f(A) \cap f(B) = [0,1] \cap \{1\} = \{1\}$$ 

(c) Show that, for an arbitrary function $g : \mathbb{R} \rightarrow \mathbb{R}$, it is always true that
$g(A \cap B) \subseteq g(A) \cap g(B)$ for all sets $A, B \subseteq \mathbb{R}$. 

\textbf{Proof:} Let $y \in \mathbb{R}$ such that $y \in g(A \cap B)$. Then, $y = g(x)$ for some $x \in (A \cap B)$. Since $x$ is in the intersection, $x$ belongs to both A and B. Thus, $y$ belongs to both $g(A)$ and $g(B)$, i.e. $y \in g(A) \cap g(B)$. \qed\\

(d) Form and prove a conjecture about the relationship between $g(A \cup B)$ and $g(A) \cup g(B)$ for an arbitrary function $g$.

Conjecture: For an arbitrary function $g: \mathbb{R} \rightarrow \mathbb{R}$, it is always true that $g(A) \cup g(B) = g(A \cup B)$.

\textbf{Proof:} First we show that $g(A) \cup g(B) \subseteq g(A \cup B)$Let $y$ be such that $y \in g(A) \cup g(B)$. Then, $y = g(x)$ for some $x \in A$ or some $x \in B$. So $x \in A \cup B$ and therefore, $y \in g(A \cup B)$.

Now, to show that $g(A \cup B) \subseteq g(A) \cup g(B)$, consider $y \in g(A \cup B)$. Then, $y = g(x)$ for some $x \in A \cup B$. So $x$ is in A or B. Therefore $y = g(x) \in g(A) \cup g(B)$. \qed \\



\noindent \textbf{1.2.12}
Let $y_1 = 6$, and for each $n \in \mathbb{N}$ define $y_{n+1} = \frac{2y_n - 6}{3}$.

(a) Use induction to prove that the sequence satisfies $y_n > -6$ for all $n \in \mathbb{N}$.

\textbf{Proof:} Base case: $n=1$. The statement is satisfied:
$$ y_1 = 6 > -6$$

Inductive hypothesis: Suppose that the statement holds for $n$. We show that it also holds for $n+1$,
$$y_{n+1} = \frac{2y_n - 6}{3} > \frac{2(-6) - 6}{3}$$
$$  = \frac{-18}{3} $$
$$  = -6$$

So, $y_{n+1} > -6$ and we are done. \qed \\


(b) Use another induction argument to show the sequence $(y_1, y_2, y_3, \ldots)$ is decreasing.

\textbf{Proof:} We want to show that for all $n \in \mathbb{N}$,  $y_n > y_{n+1}$. 

Base case: $n=1$. We have $y_1 = 6$
$$ y_2 = \frac{2y_1 - 6}{3} = \frac{2(6) - 6}{3} = 2 < 6 = y_1$$

So the statement holds for $n=1$.

Inductive hypothesis: Suppose that the statement holds for $n$ and $y_n > y_{n+1}$. We show that it also holds for $n+1$,
$$y_{n+1} = \frac{2y_n - 6}{3} > \frac{2y_{n+1} - 6}{3} = y_{n+2}$$
Where the middle inequality arises from the inductive hypothesis. \qed \\



\noindent \textbf{1.2.13}
(a) Show how induction can be used to conclude that
$$ (A_1 \cup A_2 \cup \cdots \cup A_n)^c = A^c_1 \cap A^c_2 \cap \cdots \cap A^c_n $$
for any finite $n \in \mathbb{N}$.

\textbf{Proof:} Base case: $n=1$. Clearly,
$$ (A_1)^c = A_1^c$$

Inductive hypothesis: Suppose that the statement holds for $n$ and use this to show that it holds for $n+1$. Grouping the union of the first $n$ sets into one set and applying the usual De Morgan's Law,
$$ ((A_1 \cup A_2 \cup \cdots \cup A_n) \cup A_{n+1})^c = (A_1 \cup A_2 \cup \cdots \cup A_n)^c \cap A_{n+1}^c$$
By the I.H.,
\begin{align*}
(A_1 \cup A_2 \cup \cdots \cup A_n)^c \cap A_{n+1}^c &= (A^c_1 \cap A^c_2 \cap \cdots \cap A^c_n) \cap A_{n+1}^c \\
&= A^c_1 \cap A^c_2 \cap \cdots \cap A_n^c \cap A^c_{n+1}
\end{align*}

(b) It is tempting to appeal to induction to conclude
$$ ( \bigcup_{i=1}^{\infty} A_i )^c = \bigcap_{i=1}^{\infty} A_i^c, $$
but induction does not apply here. Induction is used to prove that a
particular statement holds for every value of $n \in \mathbb{N}$, but this does not
imply the validity of the infinite case. To illustrate this point, find an
example of a collection of sets $B_1, B_2, B_3, \ldots$ where $\bigcap_{i=1}^{n} B_i \neq \emptyset$ is true
for every $n \in \mathbb{N}$, but $\bigcap_{i=1}^{\infty} B_i \neq \emptyset$ fails.

An example of such a collection of sets is
$$B_1 = \{1,2,...\}$$
$$B_2 = \{2,3,...\}$$
$$\vdots$$

(c) Nevertheless, the infinite version of De Morgan’s Law stated in (b) is a
valid statement. Provide a proof that does not use induction.

\textbf{Proof:} Suppose that $x \in ( \bigcup_{i=1}^{\infty} A_i )^c$. Then $x \notin \bigcup_{i=1}^{\infty} A_i$, which means that for all $A_i$, $x \notin A_i$. Thus, for all $A_i$, $x \in A_i^c$, so $x \in \bigcap_{i=1}^{\infty} A_i^c$.

We have shown that $(\bigcup_{i=1}^{\infty} A_i )^c \subseteq \bigcap_{i=1}^{\infty} A_i^c$. To show the other direction, let $x \in \bigcap_{i=1}^{\infty} A_i^c$. Then for all $A_i$, $x \in A_i^c$, which means that $x \notin A_i$ for all $A_i$. This implies that $x \notin \bigcup_{i=1}^{\infty} A_i$ or rather $x \in (\bigcup_{i=1}^{\infty} A_i)^c$.

This proves that $\bigcap_{i=1}^{\infty} A_i^c \subseteq (\bigcup_{i=1}^{\infty} A_i )^c$ and completes the proof. \qed \\


\end{document}
